\documentclass[]{article}
\usepackage{lmodern}
\usepackage{amssymb,amsmath}
\usepackage{ifxetex,ifluatex}
\usepackage{fixltx2e} % provides \textsubscript
\ifnum 0\ifxetex 1\fi\ifluatex 1\fi=0 % if pdftex
  \usepackage[T1]{fontenc}
  \usepackage[utf8]{inputenc}
\else % if luatex or xelatex
  \ifxetex
    \usepackage{mathspec}
    \usepackage{xltxtra,xunicode}
  \else
    \usepackage{fontspec}
  \fi
  \defaultfontfeatures{Mapping=tex-text,Scale=MatchLowercase}
  \newcommand{\euro}{€}
\fi
% use upquote if available, for straight quotes in verbatim environments
\IfFileExists{upquote.sty}{\usepackage{upquote}}{}
% use microtype if available
\IfFileExists{microtype.sty}{%
\usepackage{microtype}
\UseMicrotypeSet[protrusion]{basicmath} % disable protrusion for tt fonts
}{}
\usepackage[margin=1in]{geometry}
\usepackage{color}
\usepackage{fancyvrb}
\newcommand{\VerbBar}{|}
\newcommand{\VERB}{\Verb[commandchars=\\\{\}]}
\DefineVerbatimEnvironment{Highlighting}{Verbatim}{commandchars=\\\{\}}
% Add ',fontsize=\small' for more characters per line
\usepackage{framed}
\definecolor{shadecolor}{RGB}{248,248,248}
\newenvironment{Shaded}{\begin{snugshade}}{\end{snugshade}}
\newcommand{\KeywordTok}[1]{\textcolor[rgb]{0.13,0.29,0.53}{\textbf{{#1}}}}
\newcommand{\DataTypeTok}[1]{\textcolor[rgb]{0.13,0.29,0.53}{{#1}}}
\newcommand{\DecValTok}[1]{\textcolor[rgb]{0.00,0.00,0.81}{{#1}}}
\newcommand{\BaseNTok}[1]{\textcolor[rgb]{0.00,0.00,0.81}{{#1}}}
\newcommand{\FloatTok}[1]{\textcolor[rgb]{0.00,0.00,0.81}{{#1}}}
\newcommand{\CharTok}[1]{\textcolor[rgb]{0.31,0.60,0.02}{{#1}}}
\newcommand{\StringTok}[1]{\textcolor[rgb]{0.31,0.60,0.02}{{#1}}}
\newcommand{\CommentTok}[1]{\textcolor[rgb]{0.56,0.35,0.01}{\textit{{#1}}}}
\newcommand{\OtherTok}[1]{\textcolor[rgb]{0.56,0.35,0.01}{{#1}}}
\newcommand{\AlertTok}[1]{\textcolor[rgb]{0.94,0.16,0.16}{{#1}}}
\newcommand{\FunctionTok}[1]{\textcolor[rgb]{0.00,0.00,0.00}{{#1}}}
\newcommand{\RegionMarkerTok}[1]{{#1}}
\newcommand{\ErrorTok}[1]{\textbf{{#1}}}
\newcommand{\NormalTok}[1]{{#1}}
\usepackage{longtable,booktabs}
\ifxetex
  \usepackage[setpagesize=false, % page size defined by xetex
              unicode=false, % unicode breaks when used with xetex
              xetex]{hyperref}
\else
  \usepackage[unicode=true]{hyperref}
\fi
\hypersetup{breaklinks=true,
            bookmarks=true,
            pdfauthor={Jan Perme},
            pdftitle={Porocilo pri predmetu Analiza podatkov s programom R},
            colorlinks=true,
            citecolor=blue,
            urlcolor=blue,
            linkcolor=magenta,
            pdfborder={0 0 0}}
\urlstyle{same}  % don't use monospace font for urls
\setlength{\parindent}{0pt}
\setlength{\parskip}{6pt plus 2pt minus 1pt}
\setlength{\emergencystretch}{3em}  % prevent overfull lines
\setcounter{secnumdepth}{0}

%%% Use protect on footnotes to avoid problems with footnotes in titles
\let\rmarkdownfootnote\footnote%
\def\footnote{\protect\rmarkdownfootnote}

%%% Change title format to be more compact
\usepackage{titling}

% Create subtitle command for use in maketitle
\newcommand{\subtitle}[1]{
  \posttitle{
    \begin{center}\large#1\end{center}
    }
}

\setlength{\droptitle}{-2em}
  \title{Porocilo pri predmetu Analiza podatkov s programom R}
  \pretitle{\vspace{\droptitle}\centering\huge}
  \posttitle{\par}
  \author{Jan Perme}
  \preauthor{\centering\large\emph}
  \postauthor{\par}
  \date{}
  \predate{}\postdate{}

\usepackage[slovene]{babel}
\usepackage{graphicx}


\begin{document}

\maketitle


\section{Izbira teme}\label{izbira-teme}

Za vzorec bomo prikazali nekaj podatkov o slovenskih obcinah.

\begin{center}\rule{0.5\linewidth}{\linethickness}\end{center}

\section{Obdelava, uvoz in cišcenje
podatkov}\label{obdelava-uvoz-in-cicenje-podatkov}

Uvozili smo podatke o obcinah v obliki CSV s statisticnega urada ter v
obliki HTML z Wikipedije. Poglejmo si zacetka obeh uvoženih
razpredelnic.

\begin{Shaded}
\begin{Highlighting}[]
\KeywordTok{kable}\NormalTok{(}\KeywordTok{head}\NormalTok{(druzine))}
\end{Highlighting}
\end{Shaded}

\begin{longtable}[c]{@{}lrrrr@{}}
\toprule
& en & dva & tri & stiri\tabularnewline
\midrule
\endhead
Ajdovšcina & 1933 & 1567 & 441 & 101\tabularnewline
Apace & 512 & 248 & 43 & 5\tabularnewline
Beltinci & 949 & 744 & 132 & 18\tabularnewline
Benedikt & 226 & 227 & 44 & 20\tabularnewline
Bistrica ob Sotli & 163 & 118 & 31 & 4\tabularnewline
Bled & 934 & 655 & 127 & 17\tabularnewline
\bottomrule
\end{longtable}

\begin{Shaded}
\begin{Highlighting}[]
\KeywordTok{head}\NormalTok{(obcine)}
\end{Highlighting}
\end{Shaded}

\begin{verbatim}
##                   Površina (kv. km) Št. preb. Št. preb. (kv. km)
## Ajdovšcina                     2452    18.651                 76
## Ankaran                          80     2.984                373
## Apace                           535     3.745                 70
## Beltinci                        623     8.650                139
## Benedikt                        241     2.330                 97
## Bistrica ob Sotli               311     1.516                 49
##                   Št. naselij Ustanovitev (leto) Pokrajina
## Ajdovšcina                 45                 NA Primorska
## Ankaran                     1               2011 Primorska
## Apace                      21               2006 Štajerska
## Beltinci                    8               1994 Prekmurje
## Benedikt                   14               1998 Štajerska
## Bistrica ob Sotli          11               1998 Štajerska
##                   Statisticna regija               Odcepitev (od obcine)
## Ajdovšcina                   Goriška                                <NA>
## Ankaran                Obalno-kraška                               Koper
## Apace                       Pomurska                      Gornja Radgona
## Beltinci                    Pomurska                       Murska Sobota
## Benedikt                   Podravska Lenart (nekoc že samostojna obcina)
## Bistrica ob Sotli          Savinjska                          Podcetrtek
\end{verbatim}

Slika \ref{fig:histogram} prikazuje porazdelitev števila naselij v
obcinah.

\begin{figure}

{\centering \includegraphics{projekt_files/figure-latex/histogram-1} 

}

\caption{Histogram <U+009A>tevila naselij v obcinah}\label{fig:histogram}
\end{figure}

\begin{center}\rule{0.5\linewidth}{\linethickness}\end{center}

\section{Analiza in vizualizacija
podatkov}\label{analiza-in-vizualizacija-podatkov}

Slika \ref{fig:zemljevid} prikazuje povprecno velikost družine za vsako
obcino.

\begin{figure}

{\centering \includegraphics{projekt_files/figure-latex/zemljevid-1} 

}

\caption{Zemljevid povprecnega <U+009A>tevila otrok v dru<U+009E>ini}\label{fig:zemljevid}
\end{figure}

\begin{center}\rule{0.5\linewidth}{\linethickness}\end{center}

\section{Napredna analiza podatkov}\label{napredna-analiza-podatkov}

Slika \ref{fig:graf} prikazuje povezavo med številom naselij in površino
obcine.

\begin{figure}

{\centering \includegraphics{projekt_files/figure-latex/graf-1} 

}

\caption{Povezava med <U+009A>tevilom naselij in povr<U+009A>ino obcine}\label{fig:graf}
\end{figure}

\begin{center}\rule{0.5\linewidth}{\linethickness}\end{center}

\end{document}
